\documentclass{article}
\usepackage[T2A]{fontenc}
\usepackage[utf8]{inputenc}
\usepackage[russian]{babel}
\usepackage{amsmath}
\usepackage{amsfonts}
\usepackage{amsthm}
\usepackage{geometry}
\usepackage{hyperref}

\geometry{left=20mm,right=20mm,top=25mm,bottom=20mm}

\newtheorem{theorem}{Theorem}
\newtheorem{df}{Definition}
\newtheorem{lemma}{Lemma}

\begin{document}

\begin{lemma}
    \label{lemma:1}
    $\forall k \in \mathbb{Z} ~ \exists f, g \in \mathbb{R}[x] :$    
    \begin{align*} 
    sinxsinkx &= f(cosx) \\
    coskx &= g(cosx) 
    \end{align*}
\end{lemma}
\begin{proof}
Докажем утверждение по индукции для $\forall n \in \mathbb{N}$.
\item[База индукции]: 
\begin{align*}
    sinxsinx &= 1 - cos^2x \\
    cos2x &= 2cos^2x - 1
\end{align*}
\item[Шаг индукции]: 
\begin{align*}

\end{align*}
\end{proof}
    


\begin{theorem}
    $\forall n \in \mathbb{Z}$ функция $x \mapsto cos(nx)$ есть многочлен от cosx.    
\end{theorem}
\begin{proof}
    Очевидное следствие леммы~\ref{lemma:1}.
\end{proof}

\end{document}