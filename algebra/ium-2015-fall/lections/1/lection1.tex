\documentclass{article}
\usepackage[T2A]{fontenc}
\usepackage[utf8]{inputenc}
\usepackage[russian]{babel}
\usepackage{amsmath}
\usepackage{amsfonts}


\begin{document}

\newcommand{\df}{\overset{df}{=}}

\tableofcontents
\pagebreak
\section{Лекция 1}
Конспект первой лекции Шабата по Алгебре из курса 2015 года.

\subsection{Системы линейных уравнений}

Запишем систему линейных уравнений с двумя неизвестными
\begin{gather}
    \begin{cases} 
        a_1 x + b_1 y  = c_1\\ 
        a_2 x + b_2 y  = c_2 
    \end{cases}
\end{gather}

Её можно решать оперируя с уравнениями 
\begin{align*}
b_2 (1) - b_1 (2) &: (a_1 b_2 - a_2 b_1 )x = c_1 b_2 - b_1 c_2 \\
a_2 (1) - a_1 (2) &: (a_2 b_1 - a_1 b_2 )y = c_1 a_2 - c_1  a_1   
\end{align*}

Обозначим 
\begin{equation*}
    \begin{vmatrix}
        a_1 & b_1 \\
        a_2 & b_2
    \end{vmatrix} \df a_1 b_2 - a_2 b_1 
\end{equation*}
что называется \textbf{определителем 2-го порядка}.
\\Теперь, при 
$
\begin{vmatrix}
    a_1 & b_1 \\
    a_2 & b_2
\end{vmatrix} \neq 0 $, получаем единственное решение

\begin{gather}
        x = \frac{
            \begin{vmatrix}
                c_1 & b_1 \\
                c_2 & b_2
            \end{vmatrix}
        }{
            \begin{vmatrix}
                a_1 & b_1 \\
                a_2 & b_2
            \end{vmatrix}
        } \\
        y = \frac{
            \begin{vmatrix}
                b_1 & c_1 \\
                b_2 & c_2
            \end{vmatrix}
        }{
            \begin{vmatrix}
                a_1 & b_1 \\
                a_2 & b_2
            \end{vmatrix}
        }
\end{gather}

\subsection{Полиномиальные уравнения: решение уравнения третьей степени}

Рассмотрим кубическое уравнение от одной переменной 
\begin{equation*}
a_3 x^3 + a_2 x^2 + a_1 x + a_0 = 0 
\end{equation*}
Заменой $x = u + \frac{a_2}{3}$ уравнение сводится к виду
\begin{equation*}
x^3 + mx + n = 0
\end{equation*}
Дальше можно использовать замену $x = u + v$ и получить
\begin{equation*}
u^3 + v^3 + (u+v)(3uv + m) + n = 0
\end{equation*}
Таким образом один из корней можно найти с помощью решения системы:

\begin{equation*}
    \begin{cases}
        u^3 + v^3 = -n \\
        u^3v^3 = \frac{m}{27}
    \end{cases}        
\end{equation*}
Решение которой равносильно решению квадратного уравнения(по теореме Виетта):
\begin{gather*}
(\lambda - u^3)(\lambda - v^3) = 0 \\
\lambda^2 + n\lambda + \frac{m}{27} = 0
\end{gather*}
Откуда получаем 
\begin{gather*}
u^3 = -\frac{n}{2} + \sqrt{\frac{n^2}{4} + \frac{m^3}{27}} \\
v^3 = -\frac{n}{2} + \sqrt{\frac{n^2}{4} - \frac{m^3}{27}}
\end{gather*}
Откуда 
\begin{equation*}
x = \sqrt[3]{-\frac{n}{2} + \sqrt{\frac{n^2}{4} + \frac{m^3}{27}}} + \sqrt[3]{-\frac{n}{2} + \sqrt{\frac{n^2}{4} - \frac{m^3}{27}}}
\end{equation*}
Что, опуская формальности, можно назвать \textbf{формулой Кардано}.

\subsection{Отступление к общему случаю}
Рассмотрим более общую систему

\begin{equation*}
    \begin{cases}
     f_1(x_0, \dots,  x_n) = 0 \\
     \dots \\
     f_m(x_0, \dots,  x_n) = 0 \\
    \end{cases}
\end{equation*}
где все $f_i$ - многочлены.  
Множество решений такой системы называется \textbf{аффинным алгебраическим многообразием}.
Алгебраические многообразия изучает наука под названием \textbf{Алгебраическая геометрия}.

    Другой важный случай: решение уравнений с целыми коэффициентами в целых числах.
Этим занимается наука \textbf{Диофантова геометрия}. С каждой системой диофантовых уравнений связан вопрос об алгоритмической разрешимости.
Который и называется \textbf{10-ой проблемой Гильберта}. J. Robinson, M.Davis, Ю.В. Матиясевич доказали, что не существует алгоритма для решения системы диофантовых уравнений в общем виде.

Другим замечательным результатом являеется то, что проблема останова для конкретного алгоритма может быть сведена к вопросу о разрешимости системы диофантовых уравнений.

Вопрос о разрешимости системы полиномиальных уравнений над $\mathbb{Q}$ остаётся открытым.
\end{document}
