\documentclass{article}
\usepackage[T2A]{fontenc}
\usepackage[utf8]{inputenc}
\usepackage[russian]{babel}
\usepackage{amsmath}
\usepackage{amsfonts}
\usepackage{geometry}
\usepackage{hyperref}
\usepackage{amsthm}
\usepackage{mathrsfs}
\usepackage{faktor}
\usepackage{amssymb}


\geometry{left=20mm,right=20mm,top=25mm,bottom=20mm}

\newtheorem{task}{Задача}
\newtheorem{theorem}{Теорема}
\newtheorem{df}{Определение}
\newtheorem{lemma}{Лемма}
\newtheorem{cq}{Следствие}
\newtheorem{statement}{Утверждение}
\newtheorem{remark}{Замечание}
\newtheorem{localdf}{Обозначение}


\newcommand{\icycle}[3]{(#1_{#2} \dots #1_{#3})}
\newcommand{\gen}[1]{\langle #1 \rangle}
\newcommand{\srange}[2]{\{#1, \dots, #2\}}
\newcommand{\irange}[3]{#1_{#2}, \dots, #1_{#3}}
\newcommand{\range}[2]{#1, \dots, #2}

\newcommand{\ad}[2]{#1#2#1^{-1}}
\newcommand{\adcycle}[4]{(#4(#1_{#2}) \dots #4(#1_{#3}))}
\newcommand{\dfeq}{\overset{df}{=}}
\newcommand{\normalin}{\trianglelefteq}
\newcommand{\subgroup}{\leq}
\newcommand{\isomorphic}{\simeq}
\newcommand{\epi}[1]{\overset{#1}{\to}}
\newcommand{\mono}[1]{\overset{#1}{\hookrightarrow}}


\newcommand{\sphere}[1]{\mathbb{S}^{#1}}
\newcommand{\quotient}[2]{#1 \diagup #2}


\newcommand{\R}{\mathbb{R}}
\newcommand{\Q}{\mathbb{Q}}
\newcommand{\N}{\mathbb{N}}
\newcommand{\Z}{\mathbb{Z}}
\newcommand{\Zn}[1]{\quotient{\Z}{#1\Z}}
\newcommand{\Complex}{\mathbb{C}}
\newcommand{\class}[1]{\overline{#1}}



\begin{document}

\section*{Задача 1.}
\begin{task}
Найдите все нормальные подгруппы группы перестановок $S_4$.
\end{task}
\begin{lemma}
 Каждая перестановка из $S_n$ разлагается в произведение непересекающихся циклов единственным образом.   
\end{lemma}
\begin{proof}
    Пусть $\sigma \in S_n$. 
    Рассмотрим группу $H = \gen{\sigma} $.
    Подействуем $H$ на $\srange{1}{n}$  посредством применения функции из $H$ к элементам из $\srange{1}{n}$.
    Орбитами при этом действия будут подмножества вида 
    $Hk = \{ \sigma^i(k) | ~ \forall i \in \srange{1}{n},  k \in \srange{1}{n} \}$ 
    Так как орбиты - это разбиение множества ${\srange{1}{n}}$,
    существуют элементы ${\irange{k}{1}{r}} \in \srange{1}{n}$ такие, что 
    $\range{1}{n} = \sqcup Hk_i$.
    Значит $\forall x \in \srange{1}{n} ~ \sigma(x) = \sigma^i(k_j) $ для некоторых $i$ и $j$.
    Поэтому $\sigma = (\sigma(k_1)\sigma^2(k_1) \dots \sigma^{i_1}(k_1))\dots(\sigma(k_r)\sigma^2(k_r) \dots \sigma^{i_r}(k_r))$.
\end{proof}

\begin{df}
\textbf{Циклическим типом перестановки} называется упорядоченный по убыванию набор длин циклов в её разложении на непересекающиеся циклы.    
\end{df}

\begin{lemma}
    \label{sec1:action_by_conjugation}
    Пусть $\tau \in S_n$ и $\tau = \icycle{i}{1}{k_1} \dots \icycle{i}{1}{k_p}$ - 
    разложение $\tau$ в произведение не пересекающихся циклов.
    Пусть $\sigma \in S_n$.
    Тогда $\ad{\sigma}{\tau} = \adcycle{i}{1}{k_1}{\sigma} \dots \adcycle{i}{1}{k_p}{\sigma}$.
\end{lemma}
\begin{proof}
    Так как $Ad_{\sigma}$ - это автоморфизм $S_n$, 
    нам достаточно доказать, что $\ad{\sigma}{\icycle{i}{1}{n}} = \adcycle{i}{1}{n}{\sigma}$.
    Пусть $x \in \srange{1}{n}$,
    тогда возможны два варианта
    \begin{align*}
        x &= \sigma(i_k), \text{для некоторого} ~ k. \\ 
        x &= \sigma(y), \text{где} ~ y \notin \irange{i}{1}{n} 
    \end{align*} 
    Разберём сначала первый. Пусть $x = \sigma(i_k), \text{для некоторого} ~ k.$ 
    $\ad{\sigma}{\tau}(x) = \ad{\sigma}{\tau}(\sigma(i_k)) = \sigma(i_{k+1 ~ mod ~ n})$ \\
    Теперь второй. Пусть $x = \sigma(y), \text{где} ~ y \notin \irange{i}{1}{n}$, тогда 
    $\ad{\sigma}{\tau}(\sigma(y)) = \sigma(y)$.
\end{proof}

\begin{cq}
$S_n$ сопряжением действует транзитивно на циклических типах.
\end{cq}
Теперь можно перейти к решению задачи. 
Предположим, что $H \normalin S_4$.
Пусть $(ij) \in H$. Тогда $H = S_n$, так как транспозиции порождают $S_4$.
Значит $H$ не содержит транспозиций. 
Пусть $(ijk) \in H$. Тогда $H$ содержит все 3-циклы и кроме того содержит  
$(123)(234) = (12)(34)$, а значит и содержит все элементы с цикличиским типом $(2, 2)$.
Так как, $(ijk) = (ij)(jk)$ все 3-циклы лежат в $A_4$. 
Значит группа порождённая всеми 3-циклами содержится в $A_4$.
Однако элементов циклического типа $(2, 2)$ и $(3, 1)$ - $11$. Вместе с $Id$ - 12.
А значит $H = A_4$. Так как $[A_4: S_4] = 2$, $A_4$ - нормальна.
Теперь, пусть $H$ содержит элементы циклического типа $(2, 2)$. $H \subgroup A_4$. 
Из леммы \ref{sec1:action_by_conjugation} очевидно следует, что подгруппа состоящая из элементов циклического типа $(2, 2)$ и $Id$ нормальна.

Таким образом, мы получили, что в $S_4$ только две нормальных подгруппы: подгруппа порождённая элементами циклического типа $(2, 2)$ и $A_4$.

\begin{localdf}
Обозначим нормальную подгруппу из 4 элементов как $V_4$ 
\end{localdf}

\section*{Задача 2}

\begin{task}
Докажите, что $Z_{S_n} = {e}$ при $n \geq 3$.
\end{task}

\begin{df}
    $Пусть H \subset G$, тогда $C_G(g) \dfeq \{g \in G ~ | ~ \ad{g}{h} = h ~ \forall h \in H\}$
\end{df}

\begin{lemma}
Пусть G конечная группа и действует на себе сопряжением.
Тогда $\#Gh = [G: C_{G}(h)]$ 
\end{lemma}

\begin{proof}
Пусть $g \in \tau C_{G}(h)$, тогда $\ad{g}{h} = \ad{(\tau h_k)}{h} = \ad{\tau}{h}$.
То есть сопряжение элементами из одного смежного класса переводит $h$ в один тот же элемент.
Теперь пусть $\ad{g_1}{h} = \ad{g_2}{h}$, что равносильно тому, что $\ad{g_2^{-1}g_1}{h} = h$, 
т.е. $g_1 = g_2 (g_2^{-1} g_1)$, где $g_2^{-1} g_1 \in C_{G}(h)$, т.е. $g_1 \in g_2 C_{G}(h)$.
Значит, сопряжение двумя элементами равно тогда и только тогда, когда они лежат 
в одном и том же смежном классе. Таким образом определена иньективная и сюрьективная функция 
$g C_{G}(h) \mapsto \ad{g}{h}$.
\end{proof}

\begin{statement}
Пусть $\icycle{i}{1}{m} \in S_n$, тогда $[G: C_G(\icycle{i}{1}{m})] = \frac{n!}{m(n - m)!}$
\end{statement}

\begin{proof}
Посчитаем $\#S_{n}h$ при действии $S_n$ на себя сопряжением.
По лемме \ref{sec1:action_by_conjugation} $S_{n}h$ состоит из всех элементов циклического типа такого, как у $h$.
То есть в нашем случае, достаточно посчитать количество $m$-циклов.
По комбинаторным соображениям оно равно $C^{n}_{m} (m - 1)!$, 
так как мы сначала выбираем $m$ элементов из $n$-элементного множества, 
а потом рассматриваем их с точностью до всех перестановок, кроме циклических, которых ровно $m$.
\end{proof}

\begin{statement}
\label{sec2:cyclecentralizer}
Пусть $\icycle{i}{1}{m} \in S_n$, тогда $C_G(\icycle{i}{1}{m}) = \{\icycle{i}{1}{m}^k \sigma | \sigma \in S_{\srange{1}{n} - \{\irange{i}{1}{r}\}} \subgroup S_n\}$
\end{statement}

\begin{proof}
    \begin{equation*}
        \#[G: C_G(h)] = \frac{\#G}{\#C_G(h)} = \frac{n!}{\#C_G(h)} = \frac{n!}{m(n - m)!}    
    \end{equation*}
Откуда $\#C_G(h) = m (n - m)!$. \\
Очевидно, что $\{\icycle{i}{1}{m}^k \sigma | \sigma \in S_{\srange{1}{n} - \{\irange{i}{1}{r}\}} \subgroup S_n\} \subset C_G(\icycle{i}{1}{m})$.
При чём $\#\{\icycle{i}{1}{m}^k \sigma | \sigma \in S_{\srange{1}{n} - \{\irange{i}{1}{r}\}} \subgroup S_n\} = m (n - m)!$
Откуда и получаем искомое равенство множеств.
\end{proof}

\begin{statement}
\label{sec2:center}
$Z_{G} = \bigcap_{g \in G}C_G(g)$
\end{statement}

Предварительная подготовка закончена, 
можно переходить собственно к доказательству утверждения в задаче. 

По утверждению \ref{sec2:center} $Z_{S_n} \subset \bigcap_{i = 1}^{n}C_{S_n}((i i+1))$
\begin{gather*}
    C_{S_n}(12) = \{(12) \sigma ~ | ~ \sigma \in S_{\srange{1}{n} - \{1, 2\}}\} \\ 
    C_{S_n}(23) = \{(23) \tau ~ | ~ \tau \in S_{\srange{1}{n} - \{2, 3\}}\} \\
    C_{S_n}(12) \cap C_{S_n}(23) = S_{\srange{1}{n} - \{1, 2, 3\}}
\end{gather*}

Продолжая аналогичным образом для $(i i+1)$ получаем $\bigcap_{i = 1}^{n}C_{S_n}((i i+1)) = S_{\srange{1}{n} - \range{1}{n}} = \{e\}$

\section*{Задача 3}

\begin{task}
    Постройте изоморфизмы 
    $\quotient{\sphere{1}}{\mu_n} \isomorphic \sphere{1}$ и 
    $\quotient{\R}{\Q} \isomorphic \quotient{\sphere{1}}{\mu}$.
\end{task}

\begin{statement}
$\sphere{1} \isomorphic \quotient{\R}{2\pi\Z}$
\end{statement}

\begin{proof}
\label{sec3:sheperehom}
$\sphere{1} \subgroup \Complex$, при чём 
$\sphere{1} = \{ e^{i \phi} ~ | ~ \phi \in [0, 2 \pi)\}$. \\
Заметим, что $e^{i \phi_1} e^{i \phi_2} = e^{i(\phi_1 + \phi_2 \mod{2\pi\Z})}$ и
$ \forall \phi \in \R ~ e^{i \phi} = e^{i (\phi \mod{2\pi\Z})}$. \\
Тогда отображение $e^{i \phi} \mapsto \phi$ является гомоморфизмом и биекцией, то
есть изоморфизмом.
\end{proof}

\begin{localdf}
    Обозначим изоморфизм из доказательства утверждения \ref{sec3:sheperehom} как $\phi$.
\end{localdf}

\begin{statement}
    \begin{gather*}
        g_{\frac{n}{2\pi}}: \quotient{\R}{2 \pi \Z} \to \quotient{\R}{n\Z} \\
        g_{\frac{n}{2\pi}}: \class{x} \mapsto \class{\frac{n}{2\pi} x}
    \end{gather*}
    g - изоморфизм абелевых групп.
\end{statement}

\begin{statement}
$\phi ( g (\mu_n) ) = \Zn{n}$
\end{statement}

\begin{proof}
\begin{align*}
\mu_n &= \{ e^{\frac{2 \pi k}{n}} \in \sphere{1} ~ | ~ k \in \range{0}{n - 1}\} \\
g(\mu_n) &= \{\class{\frac{2 \pi k}{n}} \in \quotient{\R}{2 \pi \Z} ~ | ~ k \in \range{0}{n - 1}\} \\
\phi ( g (\mu_n) ) &= \{\class{k} \in \Zn{n} \subset \quotient{\R}{n \Z} ~ | ~ k \in \range{0}{n - 1}\}
\end{align*}
\end{proof}

\begin{statement}[Третья теорема об изоморфизме]
$\quotient{A}{B} \isomorphic \quotient{(\quotient{A}{C})}{(\quotient{B}{C})}$
\end{statement}



Таким образом, 
\begin{math}
    \quotient{\sphere{1}}{\mu_n}                \isomorphic 
    \quotient{(\quotient{\R}{n\Z})}{(\Zn{n})}   \isomorphic 
    \quotient{\R}{\Z}                           \isomorphic 
    \sphere{1}
\end{math}

\begin{statement}
$\class{x} \in \quotient{\R}{\Z}$ лежит в $\quotient{\Q}{\Z}$ iff $\exists n \in \Z ~ | ~ n x \in \Z$. 
\end{statement}

\begin{proof}
    Необходимость: 
    Пусть $\class{x} \in \quotient{\Q}{\Z}$, тогда $x = \frac{p}{q}$, а значит $q x \in \Z$.
    Достаточность:
    Пусть $n x = k$, $k \in \Z$, тогда $x = \frac{k}{n} \in \Q$.
\end{proof}

\begin{remark}
Это равносильно тому, что $g_{\frac{1}{2 \pi}} (\mu) = \quotient{\Q}{\Z}$
\end{remark}

Теперь, 
\begin{math}
    \quotient{\R}{\Q} \isomorphic
    \quotient{(\quotient{\R}{\Z})}{(\quotient{\Q}{\Z})} \isomorphic
    \quotient{\sphere{1}}{\mu}
\end{math}.

\section*{Задача 4}

\begin{task}
Докажите, что группа перестановок $S_n$ порождена транспозициями вида $(i i + 1)$, где $i = \range{1}{n - 1}$.
\end{task}

По индукции: \\ 

Для $n = 2$ очевидно утверждение выполняется.\\

Пусть выполняется для $S_k$. 
Тогда заметим, что $S_{k + 1} = \gen{S_{\srange{1}{k}}, S_{\srange{2}{k + 1}}}$.
В свою очередь, по предположению индукции 
\begin{align*}
    S_{\srange{1}{k}} &= \gen{\{(i i + 1) ~ | i \in \srange{1}{k - 1}\}} \\
    S_{\srange{2}{k + 1}} &= \gen{\{(i i + 1) ~ | i \in \srange{2}{k}\}}
\end{align*}
что и доказывает шаг индукции.


\section*{Задача 5}

\begin{task}
Докажите, что при $n \geq 3$ группа $S_n$ порождена элементами $(12)$ и $(12 \dots n)$.
\end{task}

\begin{equation*}
    \ad{(12 \dots n)}{(ii+1)} = (i+2 i+3).
\end{equation*}
Отсюда получаем, что $\forall i \in \srange{1}{n - 1} ~ (i i+1) \in \gen{(12), (12 \dots n)}$.
А значит по результату предудыщей задачи $\gen{(12), (12 \dots n)} = S_n$.

\section*{Задача 6}

\begin{task}
Пусть $n \geq 3$. Докажите, что для любого $k = \range{2}{n - 1}$ в группе $S_n$ 
найдётся минимальная система из $k$ образующих.  
\end{task}

По результату предыдущей задачи $S_{\range{1}{n - 1}} = \gen{(12), (12 \dots n - 1)}$.
Тогда $\gen{(12), (12 \dots n - 1), (1n)} = S_n$, 
при чём это минимальная система образующих по построению. То есть мы выполнили задачу для $k = 2, 3$.
Таким же образом, конструкция распространяется на остальные $k$:
$S_n = \gen{(12), (12 \dots n - (k - 2)), (1 n - (k - 1)), \dots, (1 n)}$ 
и это минимальная система образующих.

\section*{Задача 7}
\begin{task}
Постройте эпиморфизм $S_4 \to S_3$ и найдите его ядро.
\end{task}
\begin{equation*}
S_3 \mono{i} S_4 \epi{\pi} \quotient{S_4}{V_4}
\end{equation*}
$i$ - вложение. $i(S_3) \cap \ker\pi = i(S_3) \cap V_4 = {e}.$
Значит $ker( \pi i ) = e$, что равносильно тому, что $\pi i$ - мономорфизм.
А так как $\#S_3 = \#\quotient{S_4}{V_4}$, $\pi i$ ещё и эпиморфизм, а значит и изоморфизм.
То есть искомый эпиморфизм $\pi(\pi i)^{-1}$ и его ядро $V_4$.
\end{document}
