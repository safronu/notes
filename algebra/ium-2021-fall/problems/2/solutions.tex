\documentclass{article}
\usepackage[T2A]{fontenc}
\usepackage[utf8]{inputenc}
\usepackage[russian]{babel}
\usepackage{amsmath}
\usepackage{amsfonts}
\usepackage{geometry}
\usepackage{hyperref}
\usepackage{amsthm}
\usepackage{amssymb}
\usepackage{mathrsfs}

\geometry{left=20mm,right=20mm,top=25mm,bottom=20mm}

\newtheorem{task}{Задача}
\newtheorem{theorem}{Теорема}
\newtheorem{df}{Определение}
\newtheorem{lemma}{Лемма}
\newtheorem{cq}{Следствие}
\newtheorem{statement}{Утверждение}

\newcommand{\icycle}[3]{(#1_{#2} \dots #1_{#3})}
\newcommand{\gen}[1]{\langle #1 \rangle}
\newcommand{\range}[2]{\{#1, \dots, #2\}}
\newcommand{\irange}[3]{#1_{#2}, \dots, #1_{#3}}
\newcommand{\ad}[2]{#1#2#1^{-1}}
\newcommand{\adcycle}[4]{(#4(#1_{#2}) \dots #4(#1_{#3}))}
\newcommand{\dfeq}{\overset{df}{=}}
\newcommand{\normalin}{\trianglelefteq}
\newcommand{\subgroup}{\leq}


\begin{document}

\section*{Задача 1.}
\begin{task}
Найдите все нормальные подгруппы группы перестановок $S_4$.
\end{task}
\begin{lemma}
 Каждая перестановка из $S_n$ разлагается в произведение непересекающихся циклов единственным образом.   
\end{lemma}
\begin{proof}
    Пусть $\sigma \in S_n$. 
    Рассмотрим группу $H = \gen{\sigma} $.
    Подействуем $H$ на $\range{1}{n}$  посредством применения функции из $H$ к элементам из $\range{1}{n}$.
    Орбитами при этом действия будут подмножества вида 
    $Hk = \{ \sigma^i(k) | ~ \forall i \in \range{1}{n},  k \in \range{1}{n} \}$ 
    Так как орбиты - это разбиение множества ${\range{1}{n}}$,
    существуют элементы ${\irange{k}{1}{r}} \in \range{1}{n}$ такие, что 
    $\range{1}{n} = \sqcup Hk_i$.
    Значит $\forall x \in \range{1}{n} ~ \sigma(x) = \sigma^i(k_j) $ для некоторых $i$ и $j$.
    Поэтому $\sigma = (\sigma(k_1)\sigma^2(k_1) \dots \sigma^{i_1}(k_1))\dots(\sigma(k_r)\sigma^2(k_r) \dots \sigma^{i_r}(k_r))$.
\end{proof}

\begin{df}
\textbf{Циклическим типом перестановки} называется упорядоченный по убыванию набор длин циклов в её разложении на непересекающиеся циклы.    
\end{df}

\begin{lemma}
    \label{sec1:action_by_conjugation}
    Пусть $\tau \in S_n$ и $\tau = \icycle{i}{1}{k_1} \dots \icycle{i}{1}{k_p}$ - 
    разложение $\tau$ в произведение не пересекающихся циклов.
    Пусть $\sigma \in S_n$.
    Тогда $\ad{\sigma}{\tau} = \adcycle{i}{1}{k_1}{\sigma} \dots \adcycle{i}{1}{k_p}{\sigma}$.
\end{lemma}
\begin{proof}
    Так как $Ad_{\sigma}$ - это автоморфизм $S_n$, 
    нам достаточно доказать, что $\ad{\sigma}{\icycle{i}{1}{n}} = \adcycle{i}{1}{n}{\sigma}$.
    Пусть $x \in \range{1}{n}$,
    тогда возможны два варианта
    \begin{align*}
        x &= \sigma(i_k), \text{для некоторого} ~ k. \\ 
        x &= \sigma(y), \text{где} ~ y \notin \irange{i}{1}{n} 
    \end{align*} 
    Разберём сначала первый. Пусть $x = \sigma(i_k), \text{для некоторого} ~ k.$ 
    $\ad{\sigma}{\tau}(x) = \ad{\sigma}{\tau}(\sigma(i_k)) = \sigma(i_{k+1 ~ mod ~ n})$ \\
    Теперь второй. Пусть $x = \sigma(y), \text{где} ~ y \notin \irange{i}{1}{n}$, тогда 
    $\ad{\sigma}{\tau}(\sigma(y)) = \sigma(y)$.
\end{proof}

\begin{cq}
$S_n$ сопряжением действует транзитивно на циклических типах.
\end{cq}
Теперь можно перейти к решению задачи. 
Предположим, что $H \normalin S_4$.
Пусть $(ij) \in H$. Тогда $H = S_n$, так как транспозиции порождают $S_4$.
Значит $H$ не содержит транспозиций. 
Пусть $(ijk) \in H$. Тогда $H$ содержит все 3-циклы и кроме того содержит  
$(123)(234) = (12)(34)$, а значит и содержит все элементы с цикличиским типом $(2, 2)$.
Так как, $(ijk) = (ij)(jk)$ все 3-циклы лежат в $A_4$. 
Значит группа порождённая всеми 3-циклами содержится в $A_4$.
Однако элементов циклического типа $(2, 2)$ и $(3, 1)$ - $11$. Вместе с $Id$ - 12.
А значит $H = A_4$. Так как $[A_4: S_4] = 2$, $A_4$ - нормальна.
Теперь, пусть $H$ содержит элементы циклического типа $(2, 2)$. $H \subgroup A_4$. 
Из леммы \ref{sec1:action_by_conjugation} очевидно следует, что подгруппа состоящая из элементов циклического типа $(2, 2)$ и $Id$ нормальна.

Таким образом, мы получили, что в $S_4$ только две нормальных подгруппы: подгруппа порождённая элементами циклического типа $(2, 2)$ и $A_4$.


\section*{Задача 2}

\begin{task}
Докажите, что $Z_{S_n} = {e}$ при $n \geq 3$.
\end{task}

\begin{df}
    $Пусть H \subset G$, тогда $C_G(g) \dfeq \{g \in G ~ | ~ \ad{g}{h} = h ~ \forall h \in H\}$
\end{df}

\begin{lemma}
Пусть G конечная группа и действует на себе сопряжением.
Тогда $\#Gh = [G: C_{G}(h)]$ 
\end{lemma}

\begin{proof}
Пусть $g \in \tau C_{G}(h)$, тогда $\ad{g}{h} = \ad{(\tau h_k)}{h} = \ad{\tau}{h}$.
\end{proof}

\begin{statement}
Пусть $\icycle{i}{1}{m} \in S_n$, тогда $[G: C_G(\icycle{i}{1}{m})] = \frac{n!}{m(n - m)!}$
\end{statement}

\begin{proof}
Посчитаем $\#S_{n}h$ при действии $S_n$ на себя сопряжением.
По лемме \ref{sec1:action_by_conjugation} $S_{n}h$ состоит из всех элементов циклического типа такого, как у $h$.
То есть в нашем случае, достаточно посчитать количество $m$-циклов.
По комбинаторным соображениям оно равно $C^{n}_{m} (m - 1)!$, 
так как мы сначала выбираем $m$ элементов из $n$-элементного множества, 
а потом рассматриваем их с точностью до всех перестановок, кроме циклических, которых ровно $m$.
\end{proof}

\begin{statement}
\label{sec2:cyclecentralizer}
Пусть $\icycle{i}{1}{m} \in S_n$, тогда $C_G(\icycle{i}{1}{m}) = \{\icycle{i}{1}{m}^k \sigma | \sigma \in S_{\range{1}{n} - \{\irange{i}{1}{r}\}} \subgroup S_n\}$
\end{statement}

\begin{proof}
    \begin{equation*}
        \#[G: C_G(h)] = \frac{\#G}{\#C_G(h)} = \frac{n!}{\#C_G(h)} = \frac{n!}{m(n - m)!}    
    \end{equation*}
Откуда $\#C_G(h) = m (n - m)!$. \\
Очевидно, что $\{\icycle{i}{1}{m}^k \sigma | \sigma \in S_{\range{1}{n} - \{\irange{i}{1}{r}\}} \subgroup S_n\} \subset C_G(\icycle{i}{1}{m})$.
При чём $\#\{\icycle{i}{1}{m}^k \sigma | \sigma \in S_{\range{1}{n} - \{\irange{i}{1}{r}\}} \subgroup S_n\} = m (n - m)!$
Откуда и получаем искомое равенство множеств.
\end{proof}

\begin{statement}
\label{sec2:center}
$Z_{G} = \bigcap_{g \in G}C_G(g)$
\end{statement}

Предварительная подготовка закончена, 
можно переходить собственно к доказательству утверждения в задаче. 

По утверждению \ref{sec2:center} $Z_{S_n} \subset \bigcap_{i = 1}^{n}C_{S_n}((i i+1)) = e$ 
при $n \geq 3$ в следствии утверждения \ref{sec2:cyclecentralizer}.

\end{document}